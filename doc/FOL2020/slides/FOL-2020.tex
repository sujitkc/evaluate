\documentclass{beamer}
\usetheme{Antibes}
\usepackage{xcolor, colortbl}
\usepackage{algorithm}
\usepackage[noend]{algpseudocode}
\usepackage{textcomp}
\usepackage{listings}
\usepackage{hyperref}
\usepackage{alltt}
\usepackage{tikz}
\usepackage{framed}
\usepackage{marvosym}
\usepackage{wasysym}
\usepackage{marvosym}
\usepackage{crayola}
\usepackage{mathpartir}
\usepackage{tabularx}
\usepackage[belowskip=-15pt,aboveskip=0pt]{caption}
\usepackage[skins]{tcolorbox}
\usepackage{multicol}
\usetikzlibrary{positioning,shapes,arrows, backgrounds, fit, shadows}
\usetikzlibrary{decorations.markings}
%\usepackage{wasysym}
%\usepackage{marvosym}
\setbeamertemplate{footline}[frame number]
%\usecolortheme{fly}
\usefonttheme{serif}

\title[Sujit]{Automated Evaluation of Programming Assignments -- from Practice to Research}
\author{Sujit Kumar Chakrabarti}
\institute{IIITB}
\date{January, 2020}

\AtBeginSection[]
{
  \begin{frame}<beamer>
    \frametitle{Outline for section \thesection}
    \tableofcontents[currentsection]
  \end{frame}
}

\begin{document}
\setbeamertemplate{navigation symbols}{}%remove navigation symbols

\maketitle

\definecolor{lightblue}{rgb}{0.8,0.93,1.0} % color values Red, Green, Blue
\definecolor{darkblue}{rgb}{0.4,0.3,1.0} % color values Red, Green, Blue
\definecolor{Blue}{rgb}{0,0,1.0} % color values Red, Green, Blue
\definecolor{darkgreen}{rgb}{0,0.7,0.2} % color values Red, Green, Blue
\definecolor{Red}{rgb}{1,0,0} % color values Red, Green, Blue
\definecolor{Pink}{rgb}{0.7,0,0.2}
\definecolor{links}{HTML}{2A1B81}
\definecolor{mydarkgreen}{HTML}{126215}
\newcommand{\highlight}[1]{{\color{Red}(#1)}}

\newcommand{\myheader}[1]{
	{\color{darkblue}
		\begin{Large}
			\begin{center}
				{#1}
			\end{center}
		\end{Large}
	}
}
\newcommand{\myminorheader}[1]{
	{\color{BrickRed}
		\begin{Large}
			{\fontfamily{\sfdefault}\selectfont\textbf{#1}}
		\end{Large}
	}
}

%\tikzstyle{input} = [coordinate]
%\tikzstyle{output} = [coordinate]


\tikzstyle{bb}=[%
      rectangle, draw=black, thick, fill=OliveGreen!30, drop shadow, align=center,
      text ragged, minimum height=2em, minimum width=2em, inner sep=6pt
]

\tikzstyle{inv}=[%
      rectangle, draw=none,  align=center,
      text ragged, minimum height=2em, minimum width=2em, inner sep=6pt
]

\tikzstyle{db}=[%
      ellipse, draw=black, thick, fill=pink, drop shadow, align=center,
      text ragged, minimum height=2em, inner sep=6pt
]

\tikzstyle{jn}=[%
      ellipse, draw=black, thick, fill=black
]

\tikzstyle{io}=[%
      trapezium, trapezium left angle=60, trapezium right angle=120, draw=black, thick, fill=brown, drop shadow,
      text ragged, minimum height=2em, minimum width=2em, inner sep=6pt, align=center
]

\tikzstyle{glio}=[%
      trapezium, trapezium left angle=60, trapezium right angle=120, draw=red, line width = 1mm, fill=brown, drop shadow,
      text ragged, minimum height=2em, minimum width=2em, inner sep=6pt
]
\tikzstyle{gl}=[%
      rectangle, draw=red, line width = 1mm, fill=lightblue, drop shadow,
      text ragged, minimum height=2em, minimum width=2em, inner sep=6pt
]

\tikzstyle{en}=[%
      rectangle, draw=black, thick, fill=none,
      text ragged, minimum height=2em, minimum width=2em, inner sep=6pt
]

\tikzstyle{sh}=[%
      rectangle, draw=gray, thick, fill=none, color = gray,
      text ragged, minimum height=2em, minimum width=2em, inner sep=6pt
]


\lstset{
%	language = Python,
	basicstyle = \ttfamily\scriptsize,
	stringstyle = \ttfamily,
	keywordstyle=\color{Blue}\bfseries,
	identifierstyle=\color{Pink},
	commentstyle=\color{darkgreen},
	frameround=tttt,
%	numbers=left
	showstringspaces=false
}

\lstdefinestyle{mycode}{
	language = Caml,
	basicstyle = \tiny\ttfamily,
	stringstyle = \color{red}\ttfamily,
	keywordstyle=\color{black}\bfseries,
	identifierstyle=\ttfamily,
	frameround=tttt,
	numbers=none,
	showstringspaces=false,
	escapeinside={(*@}{@*)}
}

\newtcolorbox{myframe}[2][]{%
  enhanced,colback=white,colframe=black,coltitle=black,
  sharp corners,boxrule=0.4pt,
  fonttitle=\itshape,
  attach boxed title to top left={yshift=-0.3\baselineskip-0.4pt,xshift=2mm},
  boxed title style={tile,size=minimal,left=0.5mm,right=0.5mm,
    colback=white,before upper=\strut},
  title=#2,#1
}

\section{An Automated Evaluation System}

% frame begin %%%%%%%%%%%%%%%%%%%%%%%%
\begin{frame}{Credits}

\begin{itemize}
\item Nikhila -- PhD student
\item Ananta -- MS/R student
\item Kapil Kalra -- summer intern, 2018
\item Manish Gupta -- Professor (IIIT-B), Google AI Research Lab
\item Many students, TAs, project/thesis students, interns
\end{itemize}
\end{frame}
% frame end %%%%%%%%%%%%%%%%%%%%%%%%

% frame begin %%%%%%%%%%%%%%%%%%%%%%%%
\begin{frame}{Why Automated Evaluation?}

\begin{itemize}
\item Online learning platforms: Coursera, Udacity, EdX ...
\item Online programming contests: ACM ICPC, HakerEarth, HackerRank, CodeChef ...
\item Introductory programming courses
\item Error prone, labour intensive, repetitive \pause -- \emph{\color{Red}Boring!}
\end{itemize}
\end{frame}
% frame end %%%%%%%%%%%%%%%%%%%%%%%%

% frame begin %%%%%%%%%%%%%%%%%%%%%%%%
\begin{frame}{Our Contribution}
{Automated Evaluation System}

\begin{itemize}
\item Automatically evaluates programming assignments using testing
\item Several human weeks $\longrightarrow$ a few seconds
\item Has enabled more frequent, deeper formative assessments with shorter feedback cycles
\end{itemize}
\end{frame}
% frame end %%%%%%%%%%%%%%%%%%%%%%%%

% frame begin %%%%%%%%%%%%%%%%%%%%%%%%
\begin{frame}{Testing}
{A Test Setup}

\begin{center}

\begin{tikzpicture}
\node[inv](i) {$I_i$};
\node[bb, right = of i](th) {Test \\ Harness};
\node[inv, above = of th](put) {Program \\ under \\ Test};
\node[circle, draw=black, right = of th](cmp) {};
\draw[-] (cmp.south west) to (cmp.north east);
\draw[-] (cmp.south east) to (cmp.north west);
\node[inv, above = of cmp](e) {$E_i$};
\node[inv, right = of cmp](res) {$R_i$ (Pass/Fail)};

\draw[-latex] (i) to (th);
\draw[-latex] (put) to (th);
\draw[-latex] (th) to node[above]{$O_i$}(cmp);
\draw[-latex] (e) to (cmp);
\draw[-latex] (cmp) to (res);

\end{tikzpicture}
\begin{multicols}{2}
\begin{tabular}{| c | c |}
\hline
$I_i$ & Test input \\
\hline
$E_i$ & Expected output \\
\hline
$O_i$ & Actual output \\
\hline
$R_i$ & Test result \\
\hline
\end{tabular}

\myminorheader{Assigning Marks:}
\begin{equation*}
M = \sum k_i R_i
\end{equation*}
\end{multicols}
\end{center}

\end{frame}
% frame end %%%%%%%%%%%%%%%%%%%%%%%%

% frame begin %%%%%%%%%%%%%%%%%%%%%%%%
\begin{frame}[fragile]{Test Case}
{Example}

\begin{lstlisting}[frame=single]
def eval_triangle_area():

  import mycode.pentagon
  import code.pentagon

  t1 = mycode.pentagon.Triangle(5, 10)
  t2 = code.pentagon.Triangle(5, 10)

  eo = t1.area()
  ao = t2.area()
  
  if(equals(ao, eo)):
    return (1, "eval_triangle_area")
  else:
    return (0, "eval_triangle_area: wrong answer")
\end{lstlisting}

\begin{center}
Test case to check if a triangle's area is computed properly
\end{center}
\end{frame}
% frame end %%%%%%%%%%%%%%%%%%%%%%%%

% frame begin %%%%%%%%%%%%%%%%%%%%%%%%
\begin{frame}{System Architecture}


\begin{center}
\begin{tabular}{c @{} c}
\begin{minipage}{0.6\textwidth}

\resizebox{!}{0.7\textheight}{
\begin{tikzpicture}
\node[bb](ev){Evaluator};

\node[bb, fill=Red!20, above left= 0.2cm of ev, yshift=1cm](ts1){Test \\ suite 1};
\draw[->, thick, BrickRed](ev) |- node[left, near start] {\rotatebox{90}{$\rightarrow W_1$}} node[above, near end] {$\rightarrow\eta_1$}(ts1);
\node[bb, fill=Red!20, below left= 0.2cm of ev, yshift=-1cm](tsn){Test \\ suite $n$};
\draw[->, thick, BrickRed](ev) |- node[above left] {\rotatebox{90}{$W_n\ \leftarrow $}} node[below, near end]{$\rightarrow\eta_n$}(tsn);

\pause
\node[bb, fill=Blue!20, above left= 0.2cm of ts1, xshift=-1cm](tc11){Test \\ case 1.1};
\draw[->, thick, BrickRed](ts1) |- node[below, near end] {$ w_{1.1} \leftarrow$} node[above, near end] {$\rightarrow M_{1.1}$}(tc11);

\node[inv, left= 1.5cm of ts1](dots2){\rotatebox{90}{...}};

\node[bb, fill=Blue!20, below left= 0.2cm of ts1, xshift=-1cm](tc1n){Test \\ case $1.m_1$};
\draw[->, thick, BrickRed](ts1) |- node[below, near end] {$w_{1.m_1} \leftarrow$} node[above, near end] {$\rightarrow M_{1.m_1}$} (tc1n);

\node[bb, fill=Blue!20, above left= 0.2cm of tsn, xshift=-1cm](tcm1){Test \\ case $n.1$};
\draw[->, thick, BrickRed](tsn) |- node[below, near end] {$ w_{n.1} \leftarrow$} node[above, near end] {$\rightarrow M_{n.1}$} (tcm1);

\node[inv, left= 1.5cm of tsn](dots3){\rotatebox{90}{...}};

\node[bb, fill=Blue!20, below left= 0.2cm of tsn, xshift=-1cm](tcmn){Test \\ case $n.m_n$};
\draw[->, thick, BrickRed](tsn) |- node[below, near end] {$w_{n.m_n} \leftarrow$} node[above, near end] {$\rightarrow M_{n.m_n}$} (tcmn);

\node[inv, left= 0.2cm of ev](dots4){\rotatebox{90}{...}};


\end{tikzpicture}
}
\end{minipage}
&
\begin{minipage}{0.5\textwidth}
$\eta = \frac{\Sigma wM}{\Sigma w}$ \\

$total = \Sigma \eta W$
\end{minipage}

\end{tabular}

\end{center}
\end{frame}
% frame end %%%%%%%%%%%%%%%%%%%%%%%%

% frame begin %%%%%%%%%%%%%%%%%%%%%%%%
\begin{frame}{AEPA System}
{Achievements}

\begin{enumerate}
\item Simple setup -- no server (except LMS)
\item Simple use -- on local machine
\item Flexibility
\item Language independent
\item All data is readily available
\item Knowledge creation
\item Software offering
\end{enumerate}
\end{frame}
% frame end %%%%%%%%%%%%%%%%%%%%%%%%

\section{Approaches to Automated Evaluation}
% frame begin %%%%%%%%%%%%%%%%%%%%%%%%
\begin{frame}{Approaches to Automated Evaluation}

\begin{itemize}
\item Testing
\item Testing + Static Analysis
\item Static Analysis
\item Testing + Static Analysis + Machine Learning 
\end{itemize}
\end{frame}
% frame end %%%%%%%%%%%%%%%%%%%%%%%%

%\subsection{Testing}
% frame begin %%%%%%%%%%%%%%%%%%%%%%%%
\begin{frame}{Testing}

\myminorheader{Advantages:}
\begin{enumerate}
\item Conceptually simple
\item Easy to automate
\end{enumerate}
\pause

\vspace{0.5cm}
\myminorheader{Issues:}
\begin{enumerate}
\item Fragility
\item Designing the comparator
\item Testing structural properties
\end{enumerate}
\end{frame}
% frame end %%%%%%%%%%%%%%%%%%%%%%%%


% frame begin %%%%%%%%%%%%%%%%%%%%%%%%
\begin{frame}[fragile]{Testing}
{Issues -- Fragility}

\begin{enumerate}
\item Parsing complex outputs -- tolerating minor variability
\item Overhead on students to understand the format and code to print output in a the specified format
\item Crashes and exceptions
\item Infinite loops
\end{enumerate}

\myminorheader{Solutions:}
\begin{enumerate}
\item Running $P$ in a different thread.
\item Wrapping $P$ in input/output script and then executing
\end{enumerate}
\end{frame}
% frame end %%%%%%%%%%%%%%%%%%%%%%%%


% frame begin %%%%%%%%%%%%%%%%%%%%%%%%
\begin{frame}[fragile]{Testing}
{Issues -- Testing Structural Properties}

\myminorheader{Examples:}
\begin{itemize}
\item Has the factorial function been implemented using recursion or loop?
\item Quicksort or mergesort?
\item ...
\end{itemize}

\end{frame}
% frame end %%%%%%%%%%%%%%%%%%%%%%%%

%\subsection{Testing + Static Analysis}

% frame begin %%%%%%%%%%%%%%%%%%%%%%%%
\begin{frame}[fragile]{Testing + Static Analysis}
\myminorheader{Basic Idea:}

To use reflection to navigate through the abstract syntax tree of the program?

\myminorheader{Example:}

\begin{lstlisting}[frame=single]
@evaluate
def eval_square_baseclass():
  import code.pentagon
  s = code.pentagon.Square(10)
  base_names = [c.__name__ for c in s.__class__.__bases__]
  if(base_names == ["RegularPolygon"]):
    return (1, "eval_square_baseclass")
  else:
    return (0, "eval_square_baseclass: wrong answer")
\end{lstlisting}


\end{frame}
% frame end %%%%%%%%%%%%%%%%%%%%%%%%


% frame begin %%%%%%%%%%%%%%%%%%%%%%%%
\begin{frame}[fragile]{Testing + Static Analysis}
{Issues}

\begin{itemize}
\item The static properties that can be tested is dependent on the depth of reflection offered by the programming language. For example, in C++, how to check if class $B$ is subclass of class $A$?
\item Some properties are not easy to test; require more full-fledged static analysis. For example, how to test if a variable used has been defined earlier or not?
\end{itemize}
\end{frame}
% frame end %%%%%%%%%%%%%%%%%%%%%%%%


\section{Static Analysis}

% frame begin %%%%%%%%%%%%%%%%%%%%%%%%
\begin{frame}[fragile]{Static Analysis}
{Basic Approach}

\begin{center}
\begin{tikzpicture}
\node[inv] (inp) {Input \\ Program};
\node[bb, right = of inp] (fe){\textsc{Front} \\ \textsc{End}};
\node[bb, right = of fe] (ver) {\textsc{Verifier}};
\node[bb, above = of ver] (xr) {\textsc{Translator}};
\node[inv, left = of xr] (q) {Question};
\node[inv, right = of ver] (res) {Pass/Fail};

\draw[->, thick, draw=Maroon] (inp) to (fe); 
\draw[->, thick, draw=Maroon] (fe) to node[above]{IR}(ver); 
\draw[->, thick, draw=Maroon] (xr) to (ver); 
\draw[->, thick, draw=Maroon] (ver) to (res); 
\draw[->, thick, draw=Maroon] (q) to (xr); 

\end{tikzpicture}

\end{center}
\end{frame}
% frame end %%%%%%%%%%%%%%%%%%%%%%%%


% frame begin %%%%%%%%%%%%%%%%%%%%%%%%
\begin{frame}[fragile]{Static Analysis}
{Issues -- Partially correct submissions}

\myminorheader{Example}

\begin{framed}
\textbf{\color{Blue}Q.} Write a program that finds the transverse of a matrix (represented as a list of lists).
\end{framed}

\begin{lstlisting}[frame=single]
m = [  [1, 2, 3],  [4, 5, 6] ]
n = []
for i in range(len(m[0])):
  row = []
  for j in range(len(m)):
    row.append(m[j][i])
  n.append(row)
print n
\end{lstlisting}

\end{frame}
% frame end %%%%%%%%%%%%%%%%%%%%%%%%

\section{Static Analysis + Machine Learning}

% frame begin %%%%%%%%%%%%%%%%%%%%%%%%
\begin{frame}[fragile]{Static Analysis + Machine Learning}
{Basic Approach}

\resizebox{\textwidth}{!} {
\begin{tikzpicture}
    \node[inv] (1)  {Reference \\ Solution};
    \node[bb]  (2) [right = of 1]   {Gold \\ Standards \\ Identification};
    \node[bb]  (3) [right = of 2]   {Approach \\ Detection};
    \node[bb] (4) [right = of 3] {Comparison, \\ Repair, \\ Mark \\ Allocation};
    \node[inv] (5) [above = of 4] {Submitted \\ Solution};
    \node[inv] (6) [right = of 4] {Marks};
    \node[bb] (simdet) [above = of 3] {Similarity \\ Detection};

    \node[inv,outer sep=0] (2a) [below = of 2] {Automated \\ Testing};
    \node[inv,outer sep=0] (3a) [below = of 3]   {Unsupervised \\ Learning};
    \node[inv,outer sep=0] (4a) [below = of 4] {Supervised \\ Learning};

	\draw[->, thick, draw=Maroon] (1) -- (2);
	\draw[->, thick, draw=Maroon] (2) -- (3);
	\draw[->, thick, draw=Maroon] (3) -- (4);
	\draw[->, thick, draw=Maroon] (4) -- (6);
	\draw[->, thick, draw=Maroon] (5) -- (4);

	\draw[->, thick, dashed] (simdet) -- (3);
	\draw[->, thick, dashed] (simdet.310) |- (4.150);

	\draw[->, thick, draw=Maroon] (2a) -- (2);
	\draw[->, thick, draw=Maroon] (3a) -- (3);
	\draw[->, thick, draw=Maroon] (4a) -- (4);

  \end{tikzpicture}
}
\end{frame}
% frame end %%%%%%%%%%%%%%%%%%%%%%%%

\section{Wrap Up}

% frame begin %%%%%%%%%%%%%%%%%%%%%%%%
\begin{frame}[fragile]{Conclusions}
{Summary}

\begin{itemize}
\item A home-grown automated evaluation system
\item Approaches and challenges in automated evaluation
\item Static analysis
\item Static analysis + machine learning
\end{itemize}
\end{frame}
% frame end %%%%%%%%%%%%%%%%%%%%%%%%


% frame begin %%%%%%%%%%%%%%%%%%%%%%%%
\begin{frame}[fragile]{Thank you!}
\myheader{Thank you!}
\end{frame}
% frame end %%%%%%%%%%%%%%%%%%%%%%%%

\end{document}
